%!TEX option=--shell-escape
\documentclass[twoside]{article}
\IfFileExists{Rstyle.tex}{\input{Rstyle}}{}
\interdisplaylinepenalty=2500
\setlength{\oddsidemargin}{0 in}
\setlength{\evensidemargin}{0 in}
\setlength{\topmargin}{-0.6 in}
\setlength{\textwidth}{6.5 in}
\setlength{\textheight}{8.5 in}
\setlength{\headsep}{0.75 in}
\setlength{\parindent}{0 in}
\setlength{\parskip}{0.1 in}

\def\cmdPath{\detokenize{ knitr2tex.py }}
% Arguments: source code path, output path, start line, end line
% The fifth argument contain all other options
% the first option is to choose show souce and run (1), or only show source (0)
% the second option gives the first displayed line
\def\knitr#1#2#3#4#5{
    \immediate\write18{python \cmdPath \detokenize{#1 #2 #3 #4 #5}}
    \input{#2}
}
% Smart knitr first checks if the sub-tex file exists, if not then run knitr
\def\smartknitr#1#2#3#4#5{
    \IfFileExists{#2}{\input{#2}}{\knitr{#1}{#2}{#3}{#4}{#5}}
}

\begin{document}
Embed the source code and its output of the R code
\begin{verbatim}
\smartknitr{R_source}{tex_output}{start_line}{end_line}{1}
\end{verbatim}
\smartknitr{test.R}{sub1.tex}{1}{15}{1}

Embed the source code and its output of the R code, but omit first 4 lines, the source code start display from the fifth line.
Sometimes you need to use functions from third-party packages, but if you set the start line below the library loading command, knitr will throw an error.
This is useful when you do not want to show the library loading lines.
\begin{verbatim}
\smartknitr{R_source}{tex_output}{start_line}{end_line}{1 first_display_line}
\end{verbatim}
\smartknitr{test.R}{sub2.tex}{1}{15}{1 5}

If you only want to show several lines without evaluating them, use the option below
\begin{verbatim}
\smartknitr{R_source}{tex_output}{start_line}{end_line}{0 (option)first_display_line}
\end{verbatim}
\smartknitr{test.R}{sub3.tex}{1}{15}{0 5}
\end{document}